%% forked from https://gits-15.sys.kth.se/giampi/kthlatex kthlatex-0.2rc4 on 2020-02-13
%% expanded upon by Gerald Q. Maguire Jr.
%% This template has been adapted by Anders Sjögren to the University
%% Engineering Program in Computer Science at KTH ICT. This adaptation was
%% translation of English headings into Swedish with the addition of Swedish.
%% Many thanks to others who have provided constructive input regarding the template.

% Make it possible to conditionally depend on the TeX engine used
\RequirePackage{ifxetex}
\RequirePackage{ifluatex}
\newif\ifxeorlua
\ifxetex\xeorluatrue\fi
\ifluatex\xeorluatrue\fi

\ifxeorlua
% The following is to ensure that the PDF uses a recent version rather than the typical PDF 1-5
%  This same version of PDF should be set as an option for hyperef

\RequirePackage{expl3}
\ExplSyntaxOn

\ExplSyntaxOff
\else
\RequirePackage{expl3}
\ExplSyntaxOn
\pdf_version_gset:n{1.5}
\ExplSyntaxOff
\fi

%% Define a pair of commands to disable and reenable specific packages - see https://tex.stackexchange.com/questions/39415/unload-a-latex-package
\makeatletter
\newcommand{\disablepackage}[2]{%
  \disable@package@load{#1}{#2}%
}
\newcommand{\reenablepackage}[1]{%
  \reenable@package@load{#1}%
}
\makeatother
\ifxeorlua
\disablepackage{transparent}{}
\fi

%% The template is designed to handle a thesis in English or Swedish
% set the default language to english or swedish by passing an option to the documentclass - this handles the inside title page
% To optimize for digital output (this changes the color palette add the option: digitaloutput
% To use \ifnomenclature add the option nomenclature
% To use bibtex or biblatex - include one of these as an option
\documentclass[nomenclature, english, bibtex]{kththesis}
%\documentclass[swedish, biblatex]{kththesis}
% if pdflatex \usepackage[utf8]{inputenc}

%% Conventions for todo notes:
% Informational
%% \generalExpl{Comments/directions/... in English}
\newcommand*{\generalExpl}[1]{\todo[inline]{#1}}                

% Language-specific information (currently in English or Swedish)
\newcommand*{\engExpl}[1]{\todo[inline, backgroundcolor=kth-lightgreen40]{#1}} %% \engExpl{English descriptions about formatting}
\newcommand*{\sweExpl}[1]{\todo[inline, backgroundcolor=kth-lightblue40]{#1}}  %% % \sweExpl{Text på svenska}

% warnings
\newcommand*{\warningExpl}[1]{\todo[inline, backgroundcolor=kth-lightred40]{#1}} %% \warningExpl{warnings}

\newcommand{\mycomment}[1]{}

% Uncomment to hide specific comments, to hide **all** ToDos add `final` to
% document class
% \renewcommand\warningExpl[1]{}
% \renewcommand\generalExpl[1]{}
% \renewcommand\engExpl[1]{}
% For example uncommenting the following line hides the Swedish language explanations
% \renewcommand\sweExpl[1]{}

% The line(s) below are for BibTeX
\bibliographystyle{bibstyle/myIEEEtran}
%\bibliographystyle{apalike}


% include a variety of packages that are useful
\input{lib/includes}
\input{lib/kthcolors}
\input{lib/defines}  % load some additional definitions to make writing more consistent

% The following is needed in conjunction with generating the DiVA data with abstracts and keywords using the scontents package and a modified listings environment
%\usepackage{listings}   %  already included
\ExplSyntaxOn
\newcommand\typestoredx[2]{\expandafter\__scontents_typestored_internal:nn\expandafter{#1} {#2}}
\ExplSyntaxOff
\makeatletter
\let\verbatimsc\@undefined
\let\endverbatimsc\@undefined
\lst@AddToHook{Init}{\hyphenpenalty=50\relax}
\makeatother


\lstnewenvironment{verbatimsc}
    {
    \lstset{%
        basicstyle=\ttfamily\tiny,
        backgroundcolor=\color{white},
        %basicstyle=\tiny,
        %columns=fullflexible,
        columns=[l]fixed,
        language=[LaTeX]TeX,
        %numbers=left,
        %numberstyle=\tiny\color{gray},
        keywordstyle=\color{red},
        breaklines=true,                 % sets automatic line breaking
        breakatwhitespace=true,          % sets if automatic breaks should only happen at whitespace
        %keepspaces=false,
        breakindent=0em,
        %fancyvrb=true,
        frame=none,                     % turn off any box
        postbreak={}                    % turn off any hook arrow for continuation lines
    }
}{}

%% Add some more keywords to bring out the structure more
\lstdefinestyle{[LaTeX]TeX}{
morekeywords={begin, todo, textbf, textit, texttt}
}

%% definition of new command for bytefield package
\newcommand{\colorbitbox}[3]{%
	\rlap{\bitbox{#2}{\color{#1}\rule{\width}{\height}}}%
	\bitbox{#2}{#3}}

% define a left aligned table cell that is ragged right
\newcolumntype{L}[1]{>{\raggedright\let\newline\\\arraybackslash\hspace{0pt}}p{#1}}


\usepackage[
backref=page,
pagebackref=false,
plainpages=false,
                        % PDF related options
unicode=true,           % Unicode encoded PDF strings
bookmarks=true,         % generate bookmarks in PDF files
bookmarksopen=false,    % Do not automatically open the bookmarks in the PDF reading program
pdfpagemode=UseNone,    % None, UseOutlines, UseThumbs, or FullScreen
destlabel,              % better naming of destinations
pdfencoding=auto,       % for unicode in 
]{hyperref}
\makeatletter
\ltx@ifpackageloaded{attachfile2}{
% cannot use backref if one is using attachfile
}
{\usepackage{backref}
%
% Customize list of backreferences.
% From https://tex.stackexchange.com/a/183735/1340
\renewcommand*{\backref}[1]{}
\renewcommand*{\backrefalt}[4]{%
\ifcase #1%
        \or [Page~#2.]%
        \else [Pages~#2.]%
\fi%
}
}
\makeatother


\usepackage[all]{hypcap}	%% prevents an issue related to hyperref and caption linking

%% Acronyms
% note that nonumberlist - removes the cross references to the pages where the acronym appears
% note that super will set the descriptions text aligned
% note that nomain - does not produce a main glossary, thus only acronyms will be in the glossary
% note that nopostdot - will prevent there being a period at the end of each entry
\usepackage[acronym, style=super, section=section, nonumberlist, nomain,
nopostdot]{glossaries}
\setlength{\glsdescwidth}{0.75\textwidth}
\usepackage[]{glossaries-extra}
\ifinswedish
    %\usepackage{glossaries-swedish}
\fi

%% For use with the README_notes
% Define a new type of glossary so that the acronyms defined in the README_notes document can be distinct from those in the thesis template
% the tlg, tld, and dn will be the file extensions used for this glossary
\newglossary[tlg]{readme}{tld}{tdn}{README acronyms}


\input{lib/includes-after-hyperref}

%\glsdisablehyper
\makeglossaries
%\makenoidxglossaries

% The following bit of ugliness is because of the problems PDFLaTeX has handling a non-breaking hyphen
% unless it is converted to UTF-8 encoding.
% If you do not use such characters in your acronyms, this could be simplified to just include the acronyms file.
\ifxeorlua
\input{lib/acronyms}                %load the acronyms file
\else
\input{lib/acronyms-for-pdflatex}
\fi

% insert the configuration information with author(s), examiner, supervisor(s), ...
\input{custom_configuration}

\title{This is the title in the language of the thesis}
\subtitle{A subtitle in the language of the thesis}


\alttitle{Detta är den svenska översättningen av titeln}
\altsubtitle{Detta är den svenska översättningen av undertiteln}

% Enter the English and Swedish keywords here for use in the PDF metadata _and_ for later use
% following the respective abstract.
% Try to put the words in the same order in both languages to facilitate matching. For example:
\EnglishKeywords{Canvas Learning Management System, Docker containers, Performance tuning}
\SwedishKeywords{Canvas Lärplattform, Dockerbehållare, Prestandajustering}

%%%%% For the oral presentation
%% Add this information once your examiner has scheduled your oral presentation
\presentationDateAndTimeISO{2025-09-15 13:00}
\presentationLanguage{eng}
\presentationRoom{via Zoom https://kth-se.zoom.us/j/ddddddddddd}
\presentationAddress{Isafjordsgatan 22 (Kistagången 16)}
\presentationCity{Stockholm}

% When there are multiple opponents, separate their names with '\&'
% Opponent's information
\opponentsNames{A. B. Normal \& A. X. E. Normalè}

% Once a thesis is approved by the examiner, add the TRITA number
% The TRITA number for a thesis consists of two parts: a series (unique to each school)
% and the number in the series, which is formatted as the year followed by a colon and
% then a unique series number for the thesis - starting with 1 each year.
\trita{TRITA -- EECS-EX}{2024:0000}

% Put the title, author, and keyword information into the PDF meta information
\input{lib/pdf_related_includes}

% the custom colors and the commands are defined in defines.tex    
\hypersetup{
	colorlinks  = true,
	breaklinks  = true,
	linkcolor   = \linkscolor,
	urlcolor    = \urlscolor,
	citecolor   = \refscolor,
	anchorcolor = black
}

\ifnomenclature
% The following lines make the page numbers and equations hyperlinks in the Nomenclature list
\renewcommand*{\pagedeclaration}[1]{\unskip, \dotfill\hyperlink{page.#1}{page\nobreakspace#1}}
% The following does not work correctly, as the name of the cross-reference is incorrect
%\renewcommand*{\eqdeclaration}[1]{, see equation\nobreakspace(\hyperlink{equation.#1}{#1})}

% You can also change the page heading for the nomenclature
\renewcommand{\nomname}{List of Symbols Used}

% You can even add customization text before the list
\renewcommand{\nompreamble}{The following symbols will be later used within the body of the thesis.}
\makenomenclature
\fi

%
% The commands below are to configure JSON listings
\colorlet{punct}{red!60!black}
\definecolor{delim}{RGB}{20,105,176}
\definecolor{numb}{RGB}{106, 109, 32}
\definecolor{string}{RGB}{0, 0, 0}

\lstdefinelanguage{json}{
    numbers=none,
    numberstyle=\small,
    frame=none,
    rulecolor=\color{black},
    showspaces=false,
    showtabs=false,
    breaklines=true,
    postbreak=\raisebox{0ex}[0ex][0ex]{\ensuremath{\color{gray}\hookrightarrow\space}},
    breakatwhitespace=true,
    basicstyle=\ttfamily\small,
    extendedchars=false,
    upquote=true,
    morestring=[b]",
    stringstyle=\color{string},
    literate=
     *{0}{{{\color{numb}0}}}{1}
      {1}{{{\color{numb}1}}}{1}
      {2}{{{\color{numb}2}}}{1}
      {3}{{{\color{numb}3}}}{1}
      {4}{{{\color{numb}4}}}{1}
      {5}{{{\color{numb}5}}}{1}
      {6}{{{\color{numb}6}}}{1}
      {7}{{{\color{numb}7}}}{1}
      {8}{{{\color{numb}8}}}{1}
      {9}{{{\color{numb}9}}}{1}
      {:}{{{\color{punct}{:}}}}{1}
      {,}{{{\color{punct}{,}}}}{1}
      {\{}{{{\color{delim}{\{}}}}{1}
      {\}}{{{\color{delim}{\}}}}}{1}
      {[}{{{\color{delim}{[}}}}{1}
      {]}{{{\color{delim}{]}}}}{1}
      {’}{{\char13}}1,
}

\lstdefinelanguage{XML}
{
  basicstyle=\ttfamily\color{blue}\bfseries\small,
  morestring=[b]",
  morestring=[s]{>}{<},
  morecomment=[s]{<?}{?>},
  stringstyle=\color{black},
  identifierstyle=\color{blue},
  keywordstyle=\color{cyan},
  breaklines=true,
  postbreak=\raisebox{0ex}[0ex][0ex]{\ensuremath{\color{gray}\hookrightarrow\space}},
  breakatwhitespace=true,
  morekeywords={xmlns,version,type}% list your attributes here
}

% In case you use both listings and lstlistings - this makes them both use the same counter
\makeatletter
\AtBeginDocument{\let\c@listing\c@lstlisting}
\makeatother
\usepackage{subfiles}

% To have Creative Commons (CC) license and logos use the doclicense package
% Note that the lowercase version of the license has to be used in the modifier
% i.e., one of by, by-nc, by-nd, by-nc-nd, by-sa, by-nc-sa, zero.
% For background see:
% https://www.kb.se/samverkan-och-utveckling/oppen-tillgang-och-bibsamkonsortiet/open-access-and-bibsam-consortium/open-access/creative-commons-faq-for-researchers.html
% https://kib.ki.se/en/publish-analyse/publish-your-article-open-access/open-licence-your-publication-cc
\begin{comment}
\usepackage[
    type={CC},
    %modifier={by-nc-nd},
    %version={4.0},
    modifier={by-nc},
    imagemodifier={-eu-88x31},  % to get Euro symbol rather than Dollar sign
    hyphenation={RaggedRight},
    version={4.0},
    %modifier={zero},
    %version={1.0},
]{doclicense}
\end{comment}

\begin{document}
\selectlanguage{english}

%%% Set the numbering for the title page to a numbering series not in the preface or body
\pagenumbering{alph}
\kthcover
\clearpage\thispagestyle{empty}\mbox{} % empty back of front cover
\titlepage

% If you do not want to have a bookinfo page, comment out the line saying \bookinfopage and add a \cleardoublepage
% If you want a bookinfo page: you will get a copyright notice, unless you have used the doclicense package in which case you will get a Creative Commons license. To include the doclicense package, uncomment the configuration of this package above and configure it with your choice of license.
\bookinfopage

% Frontmatter includes the abstracts and table-of-contents
\frontmatter
\setcounter{page}{1}
\begin{abstract}
% The first abstract should be in the language of the thesis.
% Abstract fungerar på svenska också.
  \markboth{\abstractname}{}
\begin{scontents}[store-env=lang]
eng
\end{scontents}
%%% The contents of the abstract (between the begin and end of scontents) will be saved in LaTeX format
%%% and output on the page(s) at the end of the thesis with information for DiVA facilitating the correct
%%% entry of the meta data for your thesis.
%%% These page(s) will be removed before the thesis is inserted into DiVA.

\begin{scontents}[store-env=abstracts,print-env=true]
Write an abstract that is about 250 and 350 words (1/2 A4-page)  with the following components:
% key parts of the abstract
\begin{itemize}
  \item What is the topic area? (optional) Introduces the subject area for the project.
  \item Short problem statement
  \item Why was this problem worth a Bachelor's/Master’s thesis project? (\ie, why is the problem both significant and of a suitable degree of difficulty for a Bachelor's/Master’s thesis project? Why has no one else solved it yet?)
  \item How did you solve the problem? What was your method/insight?
  \item Results/Conclusions/Consequences/Impact: What are your key results/\linebreak[4]conclusions? What will others do based on your results? What can be done now that you have finished - that could not be done before your thesis project was completed?
\end{itemize}

\end{scontents}
%Text superscripts and subscripts with \textbackslash textsubscript and \textbackslash textsuperscript: A\textsubscript{x} and A\textsuperscript{x}.
%Some symbols that you might find useful are available, such as: \textbackslash textregistered, \textbackslash texttrademark, and \textbackslash textcopyright. For example, 
%the copyright symbol: \textbackslash textcopyright Maguire 2022 results in \textcopyright Maguire 2022. Additionally, here are some examples of text superscripts (which can be combined with some symbols): \textbackslash textsuperscript\{99m\}Tc, A\textbackslash textsuperscript\{*\}, A\textbackslash textsuperscript\{\textbackslash textregistered\}, and A\textbackslash texttrademark resulting in \textsuperscript{99m}Tc, A\textsuperscript{*}, A\textsuperscript{\textregistered}, and A\texttrademark. Two examples of subscripts are: H\textbackslash textsubscript\{2\}O and CO\textbackslash textsubscript\{2\} which produce  H\textsubscript{2}O and CO\textsubscript{2}.
%The following commands can be used: \textbackslash eg, \textbackslash Eg, \textbackslash ie, \textbackslash Ie, \textbackslash etc, and \textbackslash etal: \eg, \Eg, \ie, \Ie, \etc, and \etal.

\subsection*{Keywords}
\begin{scontents}[store-env=keywords,print-env=true]
% If you set the EnglishKeywords earlier, you can retrieve them with:
\InsertKeywords{english}
% If you did not set the EnglishKeywords earlier then simply enter the keywords here:
% comma separate keywords, such as: Canvas Learning Management System, Docker containers, Performance tuning
\end{scontents}
%\engExpl{\textbf{Choosing good keywords can help others to locate your paper, thesis, dissertation, \ldots and related work.}}
%Choose the most specific keyword from those used in your domain, see for example: the ACM Computing Classification System ({\small \url{https://www.acm.org/publications/computing-classification-system/how-to-use})},
%the IEEE Taxonomy ({\small \url{https://www.ieee.org/publications/services/thesaurus-thank-you.html}}), PhySH (Physics Subject Headings)\linebreak[4] ({\small \url{https://physh.aps.org/}}), \ldots or keyword selection tools such as the  National Library of Medicine's Medical Subject Headings (MeSH)  ({\small \url{https://www.nlm.nih.gov/mesh/authors.html}}) or Google's Keyword Tool ({\small \url{https://keywordtool.io/}})\\

%\textbf{Formatting the keywords}:
%\begin{itemize}
%  \item The first letter of a keyword should be set with a capital letter and proper names should be capitalized as usual.
%  \item Spell out acronyms and abbreviations.
%  \item Avoid "stop words" - as they generally carry little or no information.
%  \item List your keywords separated by commas (",").
%\end{itemize}    
%Since you should have both English and Swedish keywords - you might think of ordering them in corresponding order (\ie, so that the n\textsuperscript{th} word in each list correspond) - this makes it easier to mechanically find matching keywords.
\end{abstract}
\cleardoublepage
\babelpolyLangStart{swedish}
\begin{abstract}
    \markboth{\abstractname}{}
\begin{scontents}[store-env=lang]
swe
\end{scontents}
%\warningExpl{Inside the following scontents environment, you cannot use a \textbackslash include{filename} as it will not end up in the for diva information. Additionally, you should not use a straight double quote character in the abstracts or keywords, use two single quote characters instead.}
\begin{scontents}[store-env=abstracts,print-env=true]
%\generalExpl{Enter your Swedish abstract or summary here!}

\end{scontents}
\subsection*{Nyckelord}
\begin{scontents}[store-env=keywords,print-env=true]
% SwedishKeywords were set earlier, hence we can use alternative 2
\InsertKeywords{swedish}
\end{scontents}
\end{abstract}
\babelpolyLangStop{swedish}

\cleardoublepage

\section*{Acknowledgments}
\markboth{Acknowledgments}{}

%\engExpl{It is nice to acknowledge the people that have helped you. It is
%  also necessary to acknowledge any special permissions that you have gotten –
%  for example, getting permission from the copyright owner to reproduce a
%  figure. In this case, you should acknowledge them and this permission here
%  and in the figure’s caption. \\
%  Note: If you do \textbf{not} have the copyright owner’s permission, then you \textbf{cannot} use any copyrighted figures/tables/\ldots . Unless stated otherwise all figures/tables/\ldots are generally copyrighted.
%}

I would like to thank xxxx for having yyyy. Or in the case of two authors:\\
We would like to thank xxxx for having yyyy.

\acknowlegmentssignature

\fancypagestyle{plain}{}
\renewcommand{\chaptermark}[1]{ \markboth{#1}{}} 
\tableofcontents
  \markboth{\contentsname}{}

\cleardoublepage

%\listoffigures
%\cleardoublepage

%\listoftables
%\cleardoublepage

%\lstlistoflistings\engExpl{If you have listings in your thesis. If not, then remove this preface page.}
%\cleardoublepage

% Align the text expansion of the glossary entries
\newglossarystyle{mylong}{%
  \setglossarystyle{long}%
  \renewenvironment{theglossary}%
     {\begin{longtable}[l]{@{}p{\dimexpr 2cm-\tabcolsep}p{0.8\hsize}}}% <-- change the value here
     {\end{longtable}}%
 }
%\glsaddall
%\printglossaries[type=\acronymtype, title={List of acronyms}]
\printglossary[style=mylong, type=\acronymtype, title={List of acronyms and abbreviations}]
%\printglossary[type=\acronymtype, title={List of acronyms and abbreviations}]

%\printnoidxglossary[style=mylong, title={List of acronyms and abbreviations}]
%\engExpl{The list of acronyms and abbreviations should be in alphabetical order based on the spelling of the acronym or abbreviation.
%}

% if the nomenclature option was specified, then include the nomenclature page(s)
\ifnomenclature
    \cleardoublepage
    \printnomenclature
\fi

%% The following label is essential to know the page number of the last page of the preface
%% It is used to compute the data for the "For DIVA" pages
\label{pg:lastPageofPreface}
% Mainmatter is where the actual contents of the thesis goes
\mainmatter
\glsresetall
\renewcommand{\chaptermark}[1]{\markboth{#1}{}}
\selectlanguage{english}

\chapter{Introduction}
\label{ch:introduction}

%\engExpl{The first paragraph after a heading is not indented, all of the subsequent paragraphs have their first line indented.}
  
%This chapter describes the specific problem that this thesis addresses, the context of the problem, the
%goals of this thesis project, and outlines the structure of the thesis.\\

%\engExpl{Use the glossaries package to help yourself and your readers.
%Add the acronyms and abbreviations to lib/acronyms.tex. Some examples are shown below:}
%In this thesis, we will examine the use of \glspl{LAN}. In this thesis, we will
%assume that \glspl{LAN} include \glspl{WLAN}, such as \gls{WiFi}.

\section{Background}
\label{sec:background}

%\generalExpl{Present the background for the area. Set the context for your project – so that your reader can understand both your project and this thesis. (Give detailed background information in Chapter 2 - together with related work.)
%Sometimes it is useful to insert a system diagram here so that the reader
%knows what are the different elements and their relationship to each
%other. This also introduces the names/terms/… that you are going to use
%throughout your thesis (be consistent). This figure will also help you later
%delimit what you are going to do and what others have done or will do.}

Search and rescue (SAR) can be aided by autonomous or teleoperated robots and multi-robot systems, which can help the SAR teams map the area, monitor, or search for victims. There are multiple SAR scenarios in which robots can aid the rescuers, where one such scenario category is urban SAR in which small robots can be used to find their way through collapsed buildings or other urban environments\cite{Queralta2020}.\\

As earthquakes are very serious disasters which can be fatal due to buildings collapsing and trapping people, it is important that the victims can be found and get help fast. This is an area in which research has been done on multi-agent systems and how they can be used for collapsed building SAR scenarios.\cite{Nazarova2020}\\

My project will build on the paper by Hengstebeck, et al.\cite{Hengstebeck2024} which explores the usage of 2D boids in SAR scenarios. Their paper adds ghost boids to the boids algorithm in order to reduce collisions and direct the boids towards a target with a set strength. To reduce the amount of collisions further, Hengstebeck, et al.\cite{Hengstebeck2024} implemented a control barrier function (CBF).\\

I will begin my project with implementing the proposed boids algorithm, but in 3D, and also adding the CBF to reduce collisions. The goal is then to expand on it by adding a high level planner (HLP)\cite{HLP} in the form of an artificial potential field (APF)\cite{APF} and explore how it affects the 3D boids SAR algorithm.\\

The project is of interest to SAR organizations and the general public, as it could lead to advancements in SAR methods which help individuals in difficult scenarios. Collapsed building SAR scenarios are not that common in Sweden, as we do not have high magnitude earthquakes. Although I want to focus on these scenarios in this project, the results might be transferable to other areas that are more relevant in Sweden. Areas that this could be applicable in would be scenarios with fires in buildings or other urban SAR scenarios which might be more of interest to Swedish society.\\

The high level objective of the degree project is to contribute to the field of SAR by presenting an improvement of the methods that currently used find targets in collapsed building scenarios. The goal is to increase the efficiency of the rescuers by giving them the tools which would allow them to scout a larger area faster than they could without the tools.

\section{Problem Description}
\label{sec:problemDescription}



% Research Question
\section{Research Question}
\label{sec:researchQuestion}

\textbf{RQ1:} How does target-seeking affect the 3D boids algorithm in terms of coverage, safety, and number of boids that are able to find the target location?\cite{Hengstebeck2024}

\textbf{RQ2:} How does adding a high level planner in the form of a artificial potential field affect the 3D boids algorithm in terms of coverage, safety, and number of boids that are able to find the target.\cite{Hengstebeck2024}\cite{APF}

\subsection{Research Methodology}
\label{sec:researchMethodology}
%\generalExpl{Introduce your choice of methodology/methodologies and method/methods – and the reason why you chose them. Contrast them with and explain why you did not choose other methodologies or methods. (The details of the actual methodology and method you have chosen will be given in Chapter~\ref{ch:methods}. Note that in Chapter~\ref{ch:methods}, the focus could be research strategies, data collection, data analysis, and quality assurance.)\\
%In this section you should present your philosophical assumption(s), research method(s), and research approach(es).}

The 

\subsection{Scientific and engineering issues}

\section{Purpose}
%\generalExpl{State the purpose of your thesis and the purpose of your degree project.\\
%Describe who benefits and how they benefit if you achieve your goals. Include anticipated ethical, sustainability, social issues, etc. related to your project. (Return to these in your reflections in Section~\ref{sec:reflections}.)}

\section{Goals}
%\generalExpl{State the goal/goals of this degree project.}

%\generalExpl{In addition to presenting the goal(s), you might also state what the deliverables and results of the project are.}

The goal of the thesis is to provide insight into the potential of APF in combination with the \textit{boids algorithm} to find missing people in SAR scenarios with collapsed buildings. 

\section{Delimitations}
%\generalExpl{Describe the boundary/limits of your thesis project and what you are explicitly not going to do. This will help you bound your efforts – as you have clearly defined what is out of the scope of this thesis project. Explain the delimitations. These are all the things that could affect the study if they were examined and included in the degree project.}

Altough the goal is to provide valuable information about how well APF in combination with the boids algorithm could do in real collapsed building SAR scenarios, the focus of the project will not be to create realistic scenarios nor take into account for everything that could affect the agents in real life.

One of the delimitations for the project will be that it will not take into account how the drone would have to adjust in real life to achieve the movement in the simulation. This could lead to the agents movement not being completely realistic.

Another delimiation of the project will be that delays in signal processing will not be considered for both the communication between the different agents in the scene and the potential delays in the hardware of the real life drones. This should not make much of a difference, since the delays in todays drone hardware is quite low. %source?

\section{Structure of the thesis}
%Chapter~\ref{ch:background} presents relevant background information about xxx.  Chapter~\ref{ch:methods} presents the methodology and method used to solve the problem. …

\cleardoublepage
\chapter{Background}
\label{ch:background}

%\generalExpl{When you do your literature study, you should have a nearly complete Chapters 1 and 2.\\
%You may also find it convenient to introduce the future work section into your report early – so that you can put things that you think about but decide not to do now into this section.\\
%Note that later you can move things between this future work section and what you have done as you may change your mind about what to do now versus what to put off to future work.
%}
%\generalExpl{What does a reader (another x student -- where x is your study line) need to know to understand your report?
%What have others already done? (This is the “related work”.) Explain what and
%how prior work/prior research will be applied on or used in the degree
%project/work (described in this thesis). Explain why and what is not used in
%the degree project and give valid reasons for rejecting the work/research.}

%\section{Major background area 1}

%\subsection{Subarea 1.1}
%\subsection{Subarea 1.1.2}
%\subsection{Subarea 1.1.2}
%Using the nearest data center may improve performance, see \cite{bogdanov_nearest_2015}
%\subsection{Link layer Encapsulation}
%\label{sec:llencap}

%See Figure~\ref{fig:ieee8023-data-packet} which uses the \textsf{bytefield}  \LaTeX\ package. 

%\section{Major background area 2}

%\subsection{\glsentryshort{WLAN} Security}% you can't use the \gls() command in a heading - but you can get the short (\glsentryshort) or long version (\glsentryshort) or \glsentrylong or even the text entry (\glsentrytext) and then there is no problem - see https://tex.stackexchange.com/questions/198140/glossaries-and-custom-section-headings-broken

%\subsection{Network layer security}


%\section{Related work area}


%\subsection{Major related work 1}

%\subsection{Major related work n}

%\subsection{Minor related work 1}


\section{Summary}

%\engExpl{It is nice to have this chapter conclude with a summary. For
%  example, you can include a table that summarizes other people's ideas and
%  benefits and drawbacks with each - so as later you can compare your solution
%  to each of them. This will also help you define the variables that you will
%  use for your evaluation.}

\cleardoublepage
\chapter{Method or Methods}
\label{ch:methods}
%\generalExpl{This chapter is about Engineering-related
%  content, Methodologies and Methods.  Use a self-explaining title.\\The
%  contents and structure of this chapter will change with your choice of
%  methodology and methods.}

%\generalExpl{Describe the engineering-related contents (preferably with models) and the research methodology and methods that are used in the degree project.\\
%Give a theoretical description of the scientific or engineering methodology  you are going to use and why have you chosen this method. What other methods did you consider and why did you reject them?\\
%In this chapter, you describe what engineering-related and scientific skills you are going to apply, such as modeling, analyzing, developing, and evaluating engineering-related and scientific content. The choice of these methods should be appropriate for the problem. Additionally, you should be conscious of aspects relating to society and ethics (if applicable). The choices should also reflect your goals and what you (or someone else) should be able to do as a result of your solution - which could not be done well before you started.}


%\sweExpl{Vilka vetenskaplig eller ingenjörs-metodik ska du använda och varför har du valt den här metoden. Vilka andra metoder gjorde du övervägde du och varför du avvisar dem.
%Vad är dina mål? (Vad ska du kunna göra som ett resultat av din lösning - vilken inte kan göras i god tid innan du började)
%Vad du ska göra? Hur? Varför? Till exempel, om du har implementerat en artefakt vad gjorde du och varför? Hur kommer du utvärdera den.
%Syftet med detta kapitel är att ge en översikt över forsknings metod som
%används i denna avhandling. Avsnitt~\ref{sec:researchProcess} beskriver forskningsprocessen. Avsnitt~\ref{sec:researchParadigm} beskriver forskningsparadigmen detaljerat. Avsnitt~\ref{sec:dataCollection} fokuserar på datainsamlingstekniker som används för denna forskning. Avsnitt~\ref{sec:experimentalDesign} beskriver experimentell
%design. Avsnitt~\ref{sec:assessingReliability} förklarar de tekniker som används för att utvärdera
%tillförlitligheten och giltigheten av de insamlade uppgifterna. Avsnitt~\ref{sec:plannedDataAnalysis}
%beskriver den metod som används för dataanalysen. Slutligen, Avsnitt~\ref{sec:evaluationFramework}
%beskriver ramverket som valts för att utvärdera xxx.\\
%Ofta kan man koppla ett antal följdfrågor till undersökningsfrågan och problemlösningen t ex\\
%(1) Vilken process skall användas för konstruktion av lösningen och vilken process skall kopplas till denna för att svara på undersökningsfrågan?\\
%(2) Hur och vilket resultat (storheter) skall presenteras både för att redovisa svar på undersökningsfrågan (resultatkapitlet i denna rapport) och redovisa resultat av problemlösningen (prototypen, ofta dokument som bilagor men vilka dokument och varför?).\\
%(3) Vilken teori/teknik skall väljas och användas både för undersökningen (taxonomi, matematik, grafer, storheter mm)  och  problemlösning (UML, UseCases, Java mm) och varför?\\
%(4) Vad behöver du som student leverera för att uppnå hög kvaliet (minimikrav) eller mycket hög kvalitet på examensarbetet?\\
%(5) Frågorna kopplar till de följande underkapitlen.\\
%(6) Resonemanget bygger på att studenter på hing-programmet ofta skall konstruera något åt problemägaren och att man till detta måste koppla en intressant ingenjörsfråga. Det finns hela tiden en dualism mellan dessa aspekter i exjobbet.
%}

\section{Research Process}
\label{sec:researchProcess}

%\generalExpl{Example of using customized item labels.}
%Some steps in the process:
%\begin{enumerate}[leftmargin=*, label=\textbf{Step \arabic*}, ref=Step \arabic*] %labelindent=1em for indent
%    \itemsep0em
%    \item \label{x:s1} plan experiment,
%    \item \label{x:s2} conduct experiment,
%    \item \label{x:s3} analyze data from the experiment, and
%    \item \label{x:s4} discuss the results of the analysis.
%\end{enumerate}


\section{Research Paradigm}
\label{sec:researchParadigm}
%\sweExpl{Undersökningsparadigm\\
%Exempelvis\\
%Positivistisk (vad/hur fungerar det?) kvalitativ fallstudie med en deduktivt (förbestämd) vald ansats och ett induktivt(efterhand uppstår dataområden och data) insamlade av data och erfarenheter.}

\section{Data Collection}
\label{sec:dataCollection}
%\generalExpl{This should also show that you are aware of the social and ethical concerns that might be relevant to your data collection method.}

\subsection{Sampling}

\subsection{Sample Size}

\subsection{Target Population}

\section[Experimental design/Planned Measurements]{Experimental design and\\Planned Measurements}
\label{sec:experimentalDesign}

\subsection{Test environment/test bed/model}
%\engExpl{Describe everything that someone else would need to reproduce your test environment/test bed/model/… .}

\subsection{Hardware/Software to be used}


\section{Assessing reliability and validity of the data collected}
\label{sec:assessingReliability}

\subsection{Validity of method}
\label{sec:validtyOfMethod}
%\engExpl{How will you know if your results are valid?}
%\engExpl{Remember that validity is about the \textit{accuracy} of a measurement while reliability is about the \textit{consistency} of the measurement values under the same conditions (\ie repeatability).}

\subsection{Reliability of method}
\label{sec:reliabilityOfMethod}
%\engExpl{How will you know if your results are reliable?}

\subsection{Data validity}
\label{sec:dataValidity}
%\sweExpl{Giltigheten av uppgifter\\
%Hur vet du om dina resultat är giltiga? Är ditt resultat rättvisande?}

\subsection{Reliability of data}
\label{sec:reliabilityOfData}
%\sweExpl{Tillförlitlighet av data\\
%Hur vet du om dina resultat är tillförlitliga? Hur bra är dina resultat?}


\section{Planned Data Analysis}
\label{sec:plannedDataAnalysis}


\subsection{Data Analysis Technique}
\label{sec:dataAnalysisTechnique}

\subsection{Software Tools}
\label{sec:softwareTools}


\section{Evaluation framework}
\label{sec:evaluationFramework}

%\section{System documentation}
%\label{sec:systemDocumentation}
%\sweExpl{Systemdokumentation\\
%Med vilka dokument och hur skall en konstruerad prototyp dokumenteras? Detta blir ofta bilagor till rapporten och det som problemägaren till det ursprungliga problemet (industrin) ofta vill ha.\\
%Bland dessa bilagor återfinns ofta, och enligt någon angiven standard, kravdokument, arkitekturdokument, designdokumnet, implementationsdokument, driftsdokument, testprotokoll mm.}
%\generalExpl{If this is going to be a complete document consider putting it in as an appendix, then just put the highlights here.}


\cleardoublepage
\chapter{What you did}%\engExpl{Choose your own chapter title to describe this}
\label{ch:whatYouDid}
%\sweExpl{[Vad gjorde du? Hur gick det till? – Välj lämplig rubrik (“Genomförande”, “Konstruktion”, ”Utveckling”  eller annat]}
%\engExpl{What have you done? How did you do it? What design decisions did you make? How did what you did help you to meet your goals?}


% the following sets the TOC entry to break after the & - note you have to include the first letter of the following word as it get swolled by the \texorpdfstring{}{} processing
\section[Hardware/Software design …/Model/Simulation model \&\texorpdfstring{\\}{ p} parameters/…]{Hardware/Software design …/Model/Simulation model \& parameters/…}

\section{Implementation …/Modeling/Simulation/…}
\label{sec:implementationDetails}


%\engExpl{This section is simply to show some example of how you can include code in your thesis - this is not a section you would have in your thesis.}

%Listing~\ref{lst:helloWorldInC} shows an example of a simple program written
%in C code.

%\begin{lstlisting}[language={C}, caption={Hello world in C code}, label=lst:helloWorldInC]
%int main() {
%printf("hello, world");
%return 0;
%}
%\end{lstlisting}

%\lstset{extendedchars=true}  %% This allows characters codes in the range 128-255
%\begin{lstlisting}[language={Python}, caption={Using a python program to
%    access the KTH API to get all of the programs at KTH}, label=lst:programmes]
%KOPPSbaseUrl = 'https://www.kth.se'

%def v1_get_programmes():
%    global Verbose_Flag
%    #
%    # Use the KOPPS API to get the data
%    # note that this returns XML
%    url = "{0}/api/kopps/v1/programme".format(KOPPSbaseUrl)
%    if Verbose_Flag:
%       print("url: " + url)
%    #
%    r = requests.get(url)
%    if Verbose_Flag:
%        print("result of getting v1 programme: {}".format(r.text))
%    #
%    if r.status_code == requests.codes.ok:
%        return r.text           # simply return the XML
%    #
%    return None
%\end{lstlisting}
%\FloatBarrier

\cleardoublepage
\chapter{Results and Analysis}
\label{ch:resultsAndAnalysis}

%\engExpl{Sometimes this is split into two chapters.\\Keep in mind: How you are going to evaluate what you have done? What are your metrics?\\Analysis of your data and proposed solution\\Does this meet the goals which you had when you started?}

%In this chapter, we present the results and discuss them.


\section{Major results}

%Some statistics of the delay measurements are shown in Table~\ref{tab:delayMeasurements}.
%The delay has been computed from the time the GET request is received until the response is sent.


\section{Reliability Analysis}

\section{Validity Analysis}

\cleardoublepage
\chapter{Discussion}
\label{ch:discussion}
%\generalExpl{This can be a separate chapter or a section in the previous chapter.}

\section{RQ1:}
\label{sec:rq1}

\section{RQ2:}
\label{sec:rq2}

\section{General Research Question}
\label{sec:researchQuestion}

\section{Limitations}
\label{sec:limitations}
%\engExpl{What did you find that limited your efforts? What are the limitations of your results?}

\section{Future work}
\label{sec:futureWork}
%\engExpl{Describe valid future work that you or someone else could or should do.\\
%Consider: What you have left undone? What are the next obvious things to be done? What hints can you give to the next person who is going to follow up on your work?}

%Multi-Agent Formation Control APF

\cleardoublepage
\chapter{Conclusions and Future work}
\label{ch:conclusionsAndFutureWork}

%\generalExpl{Add text to introduce the subsections of this chapter.}

\section{Conclusions}
\label{sec:conclusions}
%\engExpl{Describe the conclusions (reflect on the whole introduction given in Chapter 1).}
  
%\engExpl{Discuss the positive effects and the drawbacks.\\
%Describe the evaluation of the results of the degree project.\\
%Did you meet your goals?\\
%What insights have you gained?\\
%What suggestions can you give to others working in this area?\\
%If you had it to do again, what would you have done differently?}

\section{Reflections}
\label{sec:reflections}
%\engExpl{What are the relevant economic, social,
%  environmental, and ethical aspects of your work?
%}

\noindent\rule{\textwidth}{0.4mm}
%\engExpl{In the references, let Zotero or other tool fill this in for you. I suggest an extended version of the IEEE style, to include URLs, DOIs, ISBNs, etc., to make it easier for your reader to find them. This will make life easier for your opponents and examiner. \\IEEE Editorial Style Manual: \url{https://www.ieee.org/content/dam/ieee-org/ieee/web/org/conferences/style_references_manual.pdf}}

\cleardoublepage
% Print the bibliography (and make it appear in the table of contents)
\renewcommand{\bibname}{References}

\phantomsection  % make it include a hyperref - see https://tex.stackexchange.com/a/98995
\addcontentsline{toc}{chapter}{References}
\bibliography{references}

%\warningExpl{If you do not have an appendix, do not include the \textbackslash cleardoublepage command below; otherwise, the last page number in the metadata will be one too large.}
\mycomment{\cleardoublepage
\appendix
\renewcommand{\chaptermark}[1]{\markboth{Appendix \thechapter\relax:\thinspace\relax#1}{}}
\chapter{Supporting materials}
\label{sec:supportingMaterial}
\generalExpl{Here is a place to add supporting material that can help others build upon your work. You can include files as attachments to the PDF file or indirectly via URLs. Alternatively, consider adding supporting material uploaded as separate files in DiVA.}

% Attach the BibTeX for your references to make it easy for a reader to find and use them
The BibTeX references used in this thesis are attached. \attachfile[description={references.bib}]{references.bib}

% Attach source code file(s) or add a URL to the github or other repository
Some source code relevant to this project can be found at \url{https://github.com/gqmaguirejr/E-learning} and \url{https://github.com/gqmaguirejr/Canvas-tools}.

An argument for including supporting material in the PDF file is that it will be available to anyone who has a copy of the PDF file. As a result, they do not have to look elsewhere for this material. This comes at the cost of a larger PDF file. However, the embedded files are encoded into a compressed stream within the PDF file; thus, reducing the number of additional bytes. For example, the references.bib file that was used in this example is \SI{10617}{\byte} in size but only occupies \SI{4261}{\byte} in the PDF file.

\warningExpl{DiVA is limited to $\approx$\SI{1}{\giga\byte} for each supporting file. If you have very large amounts of supporting material, you will probably want to use one of the data repositories. For additional help about this, contact KTH Library via 
\href{mailto:researchdata@kth.se}{researchdata@kth.se}.\\As of Spring 2024, there are plans to migrate this supporting data to 
}

\begin{figure}[!ht]
  \caption{Adobe Acrobat Reader using the paperclip icon for the attached references.bib file}
\end{figure}
\FloatBarrier

\begin{figure}[!ht]
  \caption{Adobe Acrobat Reader after right-clicking on the push-pin icon for the attached references.bib file}
\end{figure}
\FloatBarrier
\cleardoublepage

\chapter{Something Extra}
\sweExpl{svensk: Extra Material som Bilaga}

\section{Just for testing KTH colors}
\ifdigitaloutput
    \textbf{You have selected to optimize for digital output}
\else
    \textbf{You have selected to optimize for print output}
\fi
\begin{itemize}[noitemsep]
    \item Primary color
    \begin{itemize}
    \item \textcolor{kth-blue}{kth-blue \ifdigitaloutput
    actually Deep sea
    \fi} {\color{kth-blue} \rule{0.3\linewidth}{1mm} }\\

    \item \textcolor{kth-blue80}{kth-blue80} {\color{kth-blue80} \rule{0.3\linewidth}{1mm} }\\
\end{itemize}

\item  Secondary colors
\begin{itemize}[noitemsep]
    \item \textcolor{kth-lightblue}{kth-lightblue \ifdigitaloutput
    actually Stratosphere
    \fi} {\color{kth-lightblue} \rule{0.3\linewidth}{1mm} }\\

    \item \textcolor{kth-lightred}{kth-lightred \ifdigitaloutput
    actually Fluorescence\fi} {\color{kth-lightred} \rule{0.3\linewidth}{1mm} }\\

    \item \textcolor{kth-lightred80}{kth-lightred80} {\color{kth-lightred80} \rule{0.3\linewidth}{1mm} }\\

    \item \textcolor{kth-lightgreen}{kth-lightgreen \ifdigitaloutput
    actually Front-lawn\fi} {\color{kth-lightgreen} \rule{0.3\linewidth}{1mm} }\\

    \item \textcolor{kth-coolgray}{kth-coolgray \ifdigitaloutput
    actually Office\fi} {\color{kth-coolgray} \rule{0.3\linewidth}{1mm} }\\

    \item \textcolor{kth-coolgray80}{kth-coolgray80} {\color{kth-coolgray80} \rule{0.3\linewidth}{1mm} }
\end{itemize}
\end{itemize}

\textcolor{black}{black} {\color{black} \rule{\linewidth}{1mm} }

% Include an example of using nomenclature
\ifnomenclature
    \cleardoublepage
    \chapter{Main equations}
    \label{ch:NomenclatureExamples}
    This appendix gives some examples of equations that are used throughout this thesis.
    \section{A simple example}
    The following example is adapted from Figure 1 of the documentation for the package nomencl (\url{https://ctan.org/pkg/nomencl}).
    \begin{equation}\label{eq:mainEq}
    a=\frac{N}{A}
    \end{equation}
    \nomenclature{$a$}{The number of angels per unit area\nomrefeq}%       %% include the equation number in the list
    \nomenclature{$N$}{The number of angels per needle point\nomrefpage}%  %% include the page number in the list
    \nomenclature{$A$}{The area of the needle point}%
    The equation $\sigma = m a$%
    \nomenclature{$\sigma$}{The total mass of angels per unit area\nomrefeqpage}%
    \nomenclature{$m$}{The mass of one angel}
follows easily from \Cref{eq:mainEq}.

    \section{An even simpler example}
    The formula for the diameter of a circle is shown in \Cref{eq:secondEq} area of a circle in \cref{eq:thirdEq}.
    \begin{equation}\label{eq:secondEq}
    D_{circle}=2\pi r
    \end{equation}
    \nomenclature{$D_{circle}$}{The diameter of a circle\nomrefeqpage}%
    \nomenclature{$r$}{The radius of a circle\nomrefeqpage}%

    \begin{equation}\label{eq:thirdEq}
    A_{circle}=\pi r^2
    \end{equation}
    \nomenclature{$A_{circle}$}{The area of a circle\nomrefeqpage}%

    Some more text that refers to \eqref{eq:thirdEq}.
\fi  %% end of nomenclature example

\cleardoublepage

\begin{comment}
% information for examiners
\ifxeorlua
\cleardoublepage
\input{README_notes/README_examiner_notes}
\fi
\end{comment}

\begin{comment}
% Information for administrators
\ifxeorlua
\cleardoublepage
\input{README_notes/README_for_administrators.tex}
\fi
\end{comment}

\begin{comment}
% Information for Course Coordinators
\ifxeorlua
\cleardoublepage
\input{README_notes/README_for_course_coordinators}
\fi
\end{comment}

%% The following label is necessary for computing the last page number of the body of the report to include in the "For DIVA" information
\label{pg:lastPageofMainmatter}

\cleardoublepage
\clearpage\thispagestyle{empty}\mbox{} % empty page with backcover on the other side
\kthbackcover
\fancyhead{}  % Do not use header on this extra page or pages
\section*{€€€€ For DIVA €€€€}
\lstset{numbers=none} %% remove any list line numbering
\divainfo{pg:lastPageofPreface}{pg:lastPageofMainmatter}

% If there is an acronyms.tex file,
% add it to the end of the For DIVA information
% so that it can be used with the abstracts
% Note that the option "nolol" stops it from being listed in the List of Listings

% The following bit of ugliness is because of the problems PDFLaTeX has handling a non-breaking hyphen
% unless it is converted to UTF-8 encoding.
% If you do not use such characters in your acronyms, this could be simplified.
\ifxeorlua
\IfFileExists{lib/acronyms.tex}{
\section*{acronyms.tex}
\lstinputlisting[language={[LaTeX]TeX}, nolol, basicstyle=\ttfamily\color{black},
commentstyle=\color{black}, backgroundcolor=\color{white}]{lib/acronyms.tex}
}
{}
\else
\IfFileExists{lib/acronyms-for-pdflatex.tex}{
\section*{acronyms.tex}
\lstinputlisting[language={[LaTeX]TeX}, nolol, basicstyle=\ttfamily\color{black},
commentstyle=\color{black}, backgroundcolor=\color{white}]{lib/acronyms-for-pdflatex.tex}
}
{}
\fi
}

\end{document}
